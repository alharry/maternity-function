% Options for packages loaded elsewhere
\PassOptionsToPackage{unicode}{hyperref}
\PassOptionsToPackage{hyphens}{url}
%
\documentclass[
]{article}
\usepackage{lmodern}
\usepackage{setspace}
\usepackage{amssymb,amsmath}
\usepackage{ifxetex,ifluatex}
\ifnum 0\ifxetex 1\fi\ifluatex 1\fi=0 % if pdftex
  \usepackage[T1]{fontenc}
  \usepackage[utf8]{inputenc}
  \usepackage{textcomp} % provide euro and other symbols
\else % if luatex or xetex
  \usepackage{unicode-math}
  \defaultfontfeatures{Scale=MatchLowercase}
  \defaultfontfeatures[\rmfamily]{Ligatures=TeX,Scale=1}
\fi
% Use upquote if available, for straight quotes in verbatim environments
\IfFileExists{upquote.sty}{\usepackage{upquote}}{}
\IfFileExists{microtype.sty}{% use microtype if available
  \usepackage[]{microtype}
  \UseMicrotypeSet[protrusion]{basicmath} % disable protrusion for tt fonts
}{}
\makeatletter
\@ifundefined{KOMAClassName}{% if non-KOMA class
  \IfFileExists{parskip.sty}{%
    \usepackage{parskip}
  }{% else
    \setlength{\parindent}{0pt}
    \setlength{\parskip}{6pt plus 2pt minus 1pt}}
}{% if KOMA class
  \KOMAoptions{parskip=half}}
\makeatother
\usepackage{xcolor}
\IfFileExists{xurl.sty}{\usepackage{xurl}}{} % add URL line breaks if available
\IfFileExists{bookmark.sty}{\usepackage{bookmark}}{\usepackage{hyperref}}
\hypersetup{
  pdftitle={Quantifying maternal reproductive output of chondrichthyan fishes},
  hidelinks,
  pdfcreator={LaTeX via pandoc}}
\urlstyle{same} % disable monospaced font for URLs
\usepackage[margin=1in]{geometry}
\usepackage{longtable,booktabs}
% Correct order of tables after \paragraph or \subparagraph
\usepackage{etoolbox}
\makeatletter
\patchcmd\longtable{\par}{\if@noskipsec\mbox{}\fi\par}{}{}
\makeatother
% Allow footnotes in longtable head/foot
\IfFileExists{footnotehyper.sty}{\usepackage{footnotehyper}}{\usepackage{footnote}}
\makesavenoteenv{longtable}
\usepackage{graphicx}
\makeatletter
\def\maxwidth{\ifdim\Gin@nat@width>\linewidth\linewidth\else\Gin@nat@width\fi}
\def\maxheight{\ifdim\Gin@nat@height>\textheight\textheight\else\Gin@nat@height\fi}
\makeatother
% Scale images if necessary, so that they will not overflow the page
% margins by default, and it is still possible to overwrite the defaults
% using explicit options in \includegraphics[width, height, ...]{}
\setkeys{Gin}{width=\maxwidth,height=\maxheight,keepaspectratio}
% Set default figure placement to htbp
\makeatletter
\def\fps@figure{htbp}
\makeatother
\setlength{\emergencystretch}{3em} % prevent overfull lines
\providecommand{\tightlist}{%
  \setlength{\itemsep}{0pt}\setlength{\parskip}{0pt}}
\setcounter{secnumdepth}{5}
\fontsize{12}{20}
\usepackage{lineno}
\linenumbers
\linenumberfont
\renewcommand\linenumberfont{\normalfont\small\sffamily}
\usepackage{fancyhdr}
\pagestyle{fancy}
\fancyhf{}
\fancyfoot[L]{\thepage}
\renewcommand{\headrulewidth}{0pt}
\renewcommand{\footrulewidth}{0pt}
\newlength{\cslhangindent}
\setlength{\cslhangindent}{1.5em}
\newenvironment{CSLReferences}%
  {}%
  {\par}
\title{Quantifying maternal reproductive output of chondrichthyan fishes}
\usepackage{authblk}
                                        \author[1]{Alastair V Harry}
                                                            \affil{Fisheries \& Agriculture Resource Management, Department of Primary Industries and Regional Development, 39 Northside Drive, Hillarys, 6025, WA, Australia \thanks{\href{mailto:alastair.harry@gmail.com}{\nolinkurl{alastair.harry@gmail.com}}}}
                                                                                \author[2]{Ivy E Baremore}
                                                            \affil{MarAlliance, Roatán, Islas de la Bahía, Honduras}
                                                                                \author[3]{Andrew N Piercy}
                                                            \affil{College of Southern Maryland, La Plata MD 20646, United States}
                                            \date{}

\begin{document}
\maketitle

\setstretch{2}
\newpage

\textbf{Abstract}

For the live-bearing and egg-laying class of chondrichthyan fishes a three parameter logistic `maternity' function with a variable upper asymptote, \(P_{Max}\), can be used to predict the average probability of a female giving birth or laying eggs in a season. Although fundamental to calculating the reproductive capacity of a population, few studies report maternity functions, with maturity functions often used as a proxy. Applying logistic models to simulated and empirical data showed that it was feasible to estimate \(P_{Max}\) from maternal data and that precision, bias, and confidence interval coverage often improved compared to when a fixed value was used. Maturity parameters could be estimated with greater precision however substituting them for maternity parameters overestimated lifetime reproductive output. Greater use of maternity functions has the potential to improve calculation of reproductive output in quantitative populations models. In addition to improvements in parameter estimation, this method involves fewer assumptions and enables statistical inferences to be made on frequency of reproduction. This study suggests that for sample sizes \textgreater{} 100, estimation of \(P_{Max}\) is generally preferable to using a fixed value.

\textbf{Key words} Logistic regression; reproductive biology; maturity ogive; fisheries management

\newpage

\section{Introduction}\label{introduction}

Many ecological and evolutionary applications need quantitative data on the reproductive capacity of a population and in fisheries science this information forms the basis of widely-used management reference points such as spawning stock biomass (Hilborn and Walters 1992, Quinn and Deriso 1999). Calculating reproductive capacity requires data on the sexually mature proportion of the population contributing to reproduction at a given size or age. For most populations of teleost fishes a simple two-parameter logistic regression function (2PLF) is sufficient to model this process using dichotomous sexual maturity stage data (immature / mature) from a sample of individuals (Jennings et al. 2001, King 2007). While the mathematical formulae and methods of statistically estimating parameters for a 2PLF vary, the estimated values are similar and can be obtained with the built in generalised linear model (GLM) programs in most modern statistical software.

In chondrichthyan fishes (sharks, rays, and chimaeras), the process of maturation can be modelled in a similar manner to teleost fishes with a simple 2PLF; however, for the intents and purposes of calculating reproductive capacity, chondrichthyans differ in several ways to most teleosts. Firstly, in many chondrichthyans the duration of the reproductive cycle is longer than a year, meaning the proportion of females that will give birth in the population in a given season is less than one (Frisk et al. 2005, Dulvy et al. 2008, Rigby and Simpfendorfer 2015). Secondly, there may be a lag between when females reach sexual maturity and when they begin reproducing (Harry et al. 2013, Fujinami et al. 2017). This is confounded by the adopted definition of maturity, which can lead to variable estimates of size and age at maturity (Braccini et al. 2006, Walker 2007, Montealegre-Quijano et al. 2014). Finally, because female fecundity is low (usually \(\ll\) 100 embryos), total pup or egg production and recruitment are likely to be closely related to the number of reproductively active females, exhibiting little interannual variability (Taylor et al. 2013). This contrasts the often highly variable relationship between stock size and recruitment in teleost fish, and warrants a precise definition and calculation of reproductive output. For these reasons, unlike in teleosts, a maturity function may not be the most appropriate tool to quantify reproductive capacity.

\subsection{Maternity function}\label{maternity-function}

Walker (2005) recognised the need for a maternity function, as distinct from a maturity function, to accurately quantify reproductive capacity in chondrichthyans. For this purpose he used a non-linear, three-parameter logistic function (3PLF) of the form: \[Y_i \sim B(x_i; n_i, P_i)\] \[E[Y_i] = n_iP_i = \frac{\alpha}{1 + e^{-(b_0 + b_1x_i)}}\] where the proportion of individuals in a given length or age category, \(x_i\), that are in maternal condition, \(Y_i\), is a binomially distributed random variable, with an expected value equal to the probability of an individual from that category being in maternal condition, \(P_i\), multiplied by the total number of individuals in the category, \(n_i\). If \(x\) is a continuous rather than categorical variable (i.e.~\(n = 1\)), \(Y_i\) reduces to a Bernoulli random variable (Zuur et al. 2007). \(\alpha\) is the upper asymptote of the curve, controlling the maximum value of \(Y\) as \(x\) approaches infinity. The lesser-used 3PLF reduces to the ubiquitous 2PLF when \(\alpha = 1\), as is appropriate for most teleost and chondrichthyan populations if modeling maturity (Quinn and Deriso 1999, Roa et al. 1999). When \(\alpha = 1\), the function also belongs to the class of generalised linear models; logistic transformation of the dependent variable allows \(Y\) to be modeled as a linear function of \(x_i\), where \(b_0\) and \(b_1\) are the intercept and slope, respectively (Zuur et al. 2007). Alternatively, the model can be expressed in more biologically relevant terms as: \[E[Y_i] = P_{Max} \left ( 1 + e^{-ln(19)\left ( \frac{x_i - x_{50}}{x_{95} - x_{50}} \right )}  \right )^{-1} \] where \(x_{50}\) and \(x_{95}\) are the sizes or ages at which 50\% and 95\% of the maximum proportion of individuals (\(P_{Max}\)) are in maternal condition. Instead of being fit to data on female maturity condition the model is fit to dichotomous data on female maternal condition (non maternal / maternal).

Walker (2005) defines individual females as being in maternal condition if they would have given birth or laid eggs by the end of a given year such that they contribute to annual recruitment (age 0+ cohort) at the beginning of the next year. Pregnancy is a necessary, but not sufficient, condition for maternity, and the two conditions are not always synonymous. For example, reproduction in the Australian population of school shark, \emph{Galeorhinus galeus}, occurs triennially, such that approximately one third of mature females give birth annually and \(P_{Max} \approx 1/3\) (Walker 2005). Gestation lasts 20 months, and includes a protracted period of ovulation. Newly pregnant females are not considered to be in maternal condition until the year of pre-recruitment. Walker's (2005) definition is specific to species that reproduce during a single, well-defined period of the year, but the concept can be generalised to accommodate asynchronously reproducing species or those that reproduce more than once a year.

Numerous authors have adopted Walker's (2005) maternity function, including for species with biennial or longer reproductive cycles (Huveneers et al. 2007, Rochowski et al. 2015, Trinnie et al. 2016) as well as those with annual or shorter reproductive cycles (Trinnie et al. 2009, Mejía‐Falla et al. 2012, Harry et al. 2013, Taylor et al. 2016). Techniques have been described for determining maternal output in asynchronously reproducing species (Braccini et al. 2006, Colonello et al. 2016), and maternity functions are also increasingly being used directly in shark and ray population assessments (SEDAR 2012, 2017, 2023).

Despite an increase in use, only a small fraction of reproductive studies in recent years have reported maternity functions. Classification of maternal condition is more data-intensive than maturity condition, ideally requiring monthly sampling over a year or longer in order to establish the timing and duration of the female ovarian and uterine cycles. Such data can be difficult and costly to collect for sharks and rays, which are often data-poor and sampled opportunistically. The teleost-oriented foundations of fisheries science have also contributed to the general lack of awareness of maternity functions; Walker (2005) is the sole description on this type of analysis for chondrichthyans and there are no primary literature sources that describe specific methods for chondrichthyan fishes.

For practitioners that are aware of maternity functions, lack of information on implementation may also have discouraged use. Walker's (2005) approach to estimating parameters in the maternity function is difficult to reproduce, apparently due to the constraints of the proprietary statistical program first used to implement the analysis. Specifically, the method as described involves adjusting the raw data prior to parameter estimation and then weighting it during analysis (Walker 2005). Additionally, the use of a fixed value for \(P_{Max}\), as in the original analysis, has the potential to bias parameter estimates and reduce their standard errors (Motulsky and Christopoulos 2004). Alternative approaches to estimating maternal parameters have also arisen in the literature, indicative of a lack of guidance on implementation. For example, some authors have obtained maternity parameters by fitting a 2PLF to maternity data (Baremore and Hale 2012, Baremore and Passerotti 2013).

A consequence of the low uptake in use of maternity functions is that most practitioners undertaking population assessments invariably take an \emph{ad hoc} approach to quantifying maternal reproductive output. This typically involves approximating maternal output using a maturity function. Under such an approach all mature females are tacitly assumed to reproduce in each breeding season (Cortes 1998). Non-annual reproduction is accounted for by weighting fecundity or the maturity function by the assumed (but often unknown) duration of the reproductive cycle (Walker 2005). In some cases, calculations of reproductive output may use the age-at-first-reproduction, defined as the mean age at maturity plus the gestation period (Mollet and Cailliet 2002). This accounts for the protracted gestation period of many chondrichthyans, but still assumes that all individuals begin reproducing immediately after maturity. Measures of annual reproductive output that follow this approach are hence built upon a series of assumptions that are rarely tested in practice.

\subsection{Revisiting the maternity function}\label{revisiting-the-maternity-function}

Wider use of maternity functions in studies of chondrichthyan reproductive biology is needed to improve calculation of reproductive capacity in the quantitative population assessments that are increasingly being used to support fisheries management and prioritize conservation actions (Cortés et al. 2012). It may also help address the lack of empirical data on the periodicity of chondrichthyan reproductive cycles. Historically, reproductive periodicity has only been determined qualitatively, and has been assumed to be fixed (e.g.~annual, biennial), despite observations of plasticity in this trait within discrete populations (Higgs et al. 2020). Information on reproductive periodicity is contained within maternal data, however the existing approach of fixing \(P_{Max}\) in maternity functions prevents statistical inference from being made on these data. Changes to the way in which maternity functions are implemented may also contribute to quantifying uncertainty in reproductive frequency and help understand the temporal stability of reproductive cycles. To date only a single study on spiny dogfish, \emph{Squalus acanthias}, has statistically estimated \(P_{Max}\) from data (Colonello et al. 2016), and the feasibility of doing so for a wider range of species has not been investigated.

This study revisits the use of maternity functions with the objectives of providing guidance on implementation and appropriate use. To address this objective a combination of simulated and empirical data were used to 1) evaluate the performance of two alternative methods for estimating maternity parameters, subject to varying reproductive frequency, sample size, and gear selectivity, 2) illustrate the effect of not using using maternity functions on calculations of lifetime reproductive output, and 3) outline strategies for making inferences on reproductive frequency from maternity data.

\section{Materials and methods}\label{materials-and-methods}

\subsection{Approach}\label{approach}

To assess the feasibility of estimating maternity parameters a simulation study was carried out using the 3PLF to generate data for populations with varying reproductive frequencies (Figure 1). Simulations were conducted for a range of gear selectivities and sample sizes reflective of those typically available in reproductive studies and fishery sampling programs. The performance of three methods in estimating `true' maternity parameters from simulated data was tested by examining precision, bias, and interval coverage. To illustrate the effects of misspecifying maternity parameters, lifetime reproductive output was calculated for each simulated population using parameters derived from each of the methods, and by substituting maturity parameters for maternity parameters. Finally, empirical data from previously published studies were reanalysed to illustrate possible strategies for estimating maternity parameters and making inferences on reproductive frequency. All simulation modelling and data analysis was conducted using the R language {[}Version 4.2.0{]} (R Core Team 2022) and the Template Model Builder (TMB) R package {[}Version 1.9.1{]} (Kristensen et al. 2016).

\subsection{Simulation}\label{simulation}

\subsubsection{Data generation}\label{data-generation}

Data were generated for the gummy shark, \emph{Mustelus antarcticus}, and school shark, \emph{Galeorhinus galeus}, using parameters from studies conducted on southern Australian populations (Table 1) (Grant et al. 1979, Kirkwood and Walker 1986, Moulton et al. 1992, Walker 1992, Punt and Walker 1998, Walker 2005, 2007). Both species are well-studied and were chosen to be representative of fast and slow chondrichthyan life history styles, respectively (Stevens 1999).

Catch at length data were simulated using a female-only, length- and age-based equilibrium population model (Hesp 2023). The approach involved calculating expected survival and catches per recruit subject to specified life history parameters, gear selectivity, and total mortality. Growth was modelled using a von Bertalanffy growth function with length \emph{l}, of a fish of age, \emph{a}, calculated as \[l(a) = L_\infty (1- e^{-K(a-a_0)})\]

where \(L_\infty\) is asymptotic length, \emph{K} is the growth coefficient and \(a_0\) is the hypothetical age at zero length. Recruitment into the population was assumed to occur at age 0 with length conforming to a normal distribution. The proportion of fish of length, \emph{l}, at age, \(a = 0\), was calculated as \[\Theta_k = \int_{l^-_k}^{l^+_k}f_{a=0}(l)dl\] where \(l^-\) and \(l^+\) are the lower and upper limits of each 1cm length class, \emph{k}. \(f_{a=0}(l)\) is the value of the normal probability density function at age \(a = 0\) for a given length, \emph{l},\[f_{a=0}(l) = \frac{1}{\sigma\sqrt{2{\pi}}}\exp\bigg[\frac{(l-\mu)^2}{2\sigma^2}\bigg]
\]

where mean length, \(\mu\), was calculated from the growth curve. The standard deviation, \(\sigma\) was not available for either species so was assumed to be directly proportional to length, with a constant of proportionality of 5\% i.e.~\(\sigma = 0.05 \mu\) (Table 1).

Growth in subsequent age classes was modelled using a length transition matrix (Punt et al. 1997, Hall et al. 2000), \(G = g_{j,k}\) that represents the probability that a fish in length class, \(j\), will grow into length class, \(k\), over a specified time interval \[ G = \begin{bmatrix}
 g_{1,1}&0& 0 &0\\ 
 g_{2,1}& g_{2,2} & 0  &0 \\ 
 \cdots&\cdots  & \ddots  & 0\\ 
 g_{n,1}& g_{n,2} & \cdots & g_{n,n}
\end{bmatrix}\]

where the elements of the matrix followed the general form of Punt et. al. (1997) and are given by\[ g_{k,j} = \left\{
    \begin{array}{ll}
        \int_{{\infty}^-}^{\mathrm{l_{k}^+}}f[\phi(l,j)]dl & \mbox{if }k=1 \\
        \int_{l_{k}^-}^{\mathrm{l_{k}^+}}f[\phi(l,j)]dl &\mbox{if }k=1<k<n \\
\int_{l_{k}^-}^{\mathrm{\infty^+}}f[\phi(l,j)]dl &\mbox{if }k=n \\
    \end{array}
\right.
\]

where \emph{l} is mean length, \emph{f} is the specified normal distribution, \(\mathrm{l_{k}^-}\) and \(\mathrm{l_{k}^+}\) are the lower and upper limits of length class \emph{j}, and \(\phi\) is a vector of parameters.

Selectivity in the model was assumed to follow a gamma function based on gillnet selectivity experiments (Kirkwood and Walker 1986, Punt and Walker 1998), where the selectivity of an individual in length class, \(l\) was given by \[S(l) = \left [ \frac{l}{\theta_1 \theta_2} \right ]^{\theta_1 }exp(\theta_1-\frac{l}{\theta_2}) \] where \(\theta_1\) and \(\theta_2\) are estimated parameters (Kirkwood and Walker 1986). Two selectivity scenarios were simulated corresponding to low and high selectivity of the maternal component of the population. For consistency between species, gillnet mesh sizes were chosen that would result in peak relative selectivity occurring at the lengths equal to 25\% and 75\% of the population in maternal condition (Table 1).

Fishing mortality, \emph{F}, in length class \(l\), was calculated as \[F(l) = S(l) \cdot F\]where \(F\) was assumed to be constant and nominally set to 25\% of natural mortality, \(M\) (i.e.~\(0.25M\)). The effects of varying levels of \(F\) were not investigated further in this study. Total mortality, \emph{Z,} in length class, \(l\), was further given by \[Z(l) = F(l)\cdot{M} \]

The number of fish, \(N\), per recruit in length class, \(l\), that survived to age, \(a\), was calculated as

\[ N_{l,a} = \left\{    \begin{array}{ll}        1 & \mbox{if }a=0 \\        N_{l,a-1}\exp(-Z_{l}) &\mbox{if }a=0<a<A_{Max} \\        N_{l,a-1}\exp(-Z_{l})/(1-\exp(-Z_{l})) &\mbox{if }a=A_{Max} \\    \end{array}\right.\]

where \(A_{Max}\) is the maximum age (Table 1). The estimated numbers of fish caught in length class, \(l\), at age, \(a\), was calculated using the Baranov catch equation as

\[
\hat{C}_{l} = \sum_a(F_{l}/Z_{l})(1-\exp(-Z_{l}))N_{l,a}
\]

To examine the effect of varying sample size, six scenarios were conducted with catch ranging from 50 to 2500 individuals, reflecting a gradient from data poor to rich.

Next, dichotomous maturity-at-length data, \(\psi(l)\), were randomly generated by simulating from a Bernoulli distribution, \(\Psi(l) \sim B(P)\), where \(P\) was given by the 2PLF: \[P =  \left ( 1 + e^{-ln(19)\left ( \frac{l - L_{50}}{L_{95} - L_{50}} \right )}  \right )^{-1}\] This process was then repeated to generate maternity-at-length data, \(\Psi'(l)\), using the 3PLF: \[\Psi' \sim B(P)\] \[P =  P_{Max}\left ( 1 + e^{-ln(19)\left ( \frac{l - L'_{50}}{L'_{95} - L'_{50}} \right )}  \right )^{-1}\] Finally, 300 iterations were run for each of the 144 unique variables combinations.

\subsubsection{Estimation approaches}\label{estimation-approaches}

Two approaches to estimating maternity parameters were compared: the 3PLF itself (3PLF-estimated) and the 3PLF function with a fixed asymptote (3PLF-fixed). Additionally, two approaches using a 2PLF were also examined, using maternity-at-length data (2PLF-maternity) and maturity-at-length data (2PLF-maturity). The 3PLF-estimated method was undertaken to validate the utility of this model, which has so far been used in only a single study (Colonello et al. 2016). The 3PLF-fixed is the method described by Walker (2005) and most commonly used in practice. The 2PLF-maternity method was used to examine what effect simply ignoring the upper asmyptote had (i.e.~\(P_{Max}\) fixed at 1). The 2PLF-maturity method is commonly used to approximate a maternity curve, and tacitly assumed to be similar. The resulting maturity curve can then be weighted by the proportion of gravid females to calculate annual reproductive output (although in practice annual fecundity is more often modified, for example halved for a species suspected to reproduce biennially (SEDAR 2023)).

Only the 3PLF-estimated method involves statistically estimating \(P_{Max}\), which must be subjectively chosen in the case of the 3PLF-fixed and the 2PLF-maturity. \(P_{Max}\) is ideally chosen based on detailed study of the ovarian and uterine cycles (Walker 2005), or alternatively based on the proportion of mature females observed to be in maternal condition during sampling (Baremore and Hale 2012, Harry et al. 2013, Trinnie et al. 2016). For the purposes of the simulation, the timing and duration of the uterine and ovarian cycles were assumed to be unknown. The following procedure was used to `guess' the fixed value of \(P_{Max}\) in each simulation. \(P_{Max}\) was chosen as the proportion of females in maternal condition above a the length at which \(99\%\) of females were mature, \(L > \Psi(L_{99\%})\). In some cases this procedure failed due to there being no maternal females in the simulated data set meeting this criteria. If this occurred, \(L_{95}\) was used, followed by \(L_{50}\), and finally the proportion of all mature females in the sample that were in maternal condition. Statistical estimation of all non-fixed parameters in each of the methods was undertaken using maximum likelihood. Nonparametric bootstrapping was used to calculate approximate 50\% confidence intervals for the best-fit parameters from 250 resampled data sets.

\subsubsection{Reproductive output}\label{reproductive-output}

For each iteration, the per-generation rate of multiplication, \(R_0\), or lifetime female pup production, was calculated as: \[R_0 =\int_{0}^{\infty}R\beta(a)e^{\int_{0}^{a}M ds}da \] where \(R\) and \(M\) are the embryonic sex ratio and natural mortality rate (Table 1) and \(\beta(a)\) is the age-specific reproductive rate which incorporates \(\Psi'(L)\) (Xiao and Walker 2000).

\subsubsection{Performance}\label{performance}

Model performance was evaluated in each iteration where 1) the fitted model successfully converged as indicated by a positive definite Hessian matrix, and 2) the estimated parameters \(L_{50}'\) and \(L_{95}'\) remained within specified bounds (\(0 - 10^4\)). An iteration was also not attempted if there were no maternal females in the randomly generated sample. Model performance was evaluated by quantifying precision, bias, and interval coverage for \(L'_{50}\), \(P_{Max}\) and \(R_0\). Bias was quantified using the mean relative error, \(\text{MRE} = \text{mean}[(x_{\text{Estimated}} - x_{\text{True}})/x_{\text{True}}]\) and precision using the mean absolute relative error, \(\text{MARE} = \text{mean}|[(x_{\text{Estimated}} - x_{\text{True}})/x_{\text{True}}]|\). Interval coverage measures the ability of the model to capture uncertainty and was calculated by taking the proportion of the estimated parameters for each simulation that fell within the 50\% confidence intervals (Rudd and Thorson 2018). For a well-performing model, approximately 50\% of parameters would be expected to fall withing the nominal 50\% confidence intervals.

\subsubsection{Empirical case study}\label{empirical-case-study}

An empirical case study was also undertaken to show how maternity functions can be used in practice. Data from two studies of the reproductive biology of the sandbar shark, \emph{Carcharhinus plumbeus}, in the Gulf of Mexico and western North Atlantic Ocean were combined and reanalysed using both 3PLF methods. The sandbar shark is a relatively long-lived (\(A_{Max}\) = 33 years) and slow growing species (\(K = 0.12 \text{ yr}^{-1}\)) with a protracted reproductive cycle lasting longer than a year (Springer 1960). The Gulf of Mexico and western North Atlantic Ocean population has been considered overfished since the late 1970s (Sminkey and Musick 1995) and during the 2000s detailed studies of sandbar shark reproductive biology and growth were undertaken for stock assessment purposes (Baremore and Hale 2012, Piercy et al. 2016). In their study, Baremore and Hale (2012) estimated maternity parameters using the 2PLF-maternity method, weighting the final curve by 0.37 based on the proportion of pregnant females observed, most closely aligning with a triennial reproductive cycle. Piercy \emph{et al.} (2016) did not estimate maternity parameters but also suggested that the average reproductive cycle was likely to be longer than two years based on examination of ovarian follicles. To examine relative support for either a biennial or triennial reproductive cycle, 3PLF-fixed models were fit with \(P_{Max}\) estimated and fixed at 0.5 and 0.33, respectively.

\section{Results}\label{results}

\subsection{Simulation study}\label{simulation-study}

Parameter estimation was generally straightforward; with sample sizes \(> 50\) a success rate of \(\geq 95\%\) was achieved in all scenarios (Figure S1, Table S1). Overall, convergence and parameter boundary failures were most common with the 2PLF-maternity method which essentially involved fitting data to an under-parameterized model where \(P_{Max}\) was always fixed at 1. In addition to convergence failures, 12 simulations also generated insufficient maternal data (zero or one maternal females in the simulated data), and were discarded.

Generally when using the 3PLF methods model performance declined as the periodicity of the reproductive cycle increased. The nature of this problem can be seen in the positive correlation between the \(\hat{L'_{50}}\) and \(\hat{P_{Max}}\) parameters whereby \(P_{Max}\) affects the magnitude and direction of bias in these parameters (Figure 2). With lower values of \(P_{Max}\) it was apparently more difficult for the model to separate the point of inflection in the logistic model and the upper asymptote. As true underlying \(P_{Max}\) decreased there was also a greater potential for positive bias in \(\hat{P_{Max}}\) (Figure 2).

For the 3PLF methods, parameter estimates tended to be biased high at lower sample sizes (Figure S2, Figure S3). In some scenarios, particularly with sample sizes \(\leq\) 250, fixing the asymptote was effective at reducing bias, although for larger sample sizes neither method was clearly preferable (Figure S2, Figure S3). The accuracy of parameter estimates varied considerably across different combinations of variables used in the simulation and among parameters (Figure S4, Figure S5). Again, in some scenarios with smaller sample sizes \(\leq\) 250, fixing the asymptote improved precision with the 3PLF methods, although at larger sample sizes better precision was achieved by estimating \(P_{Max}\) (Figure 3). Overall, \(P_{Max}\) was comparatively more challenging to estimate accurately than \(L'_{50}\) (Figure S4, Figure S5).

While the 3PLF-fixed method was in some cases able to reduce bias at lower sample sizes, a trade off in the use of this method was poorer interval coverage (Figure 4, Figure S6). For most simulations, interval coverage for \(\hat{L'_{50}}\) was well below the expected level of 50\%, irrespective of sample size. In contrast, for the 3PLF-estimated method interval coverage oscillated around 50\% for both \(\hat{L'_{50}}\) and \(\hat{P_{Max}}\).

Differences in life history and gear selectivity played an important role in the ability to obtain precise and unbiased maternity parameters. Certain combinations of variables in the simulated data resulted in fewer females in maternal or immature condition making parameter estimation difficult (Figure S8, Figure S9). These effects were not necessarily consistent between species and seemed to reflect differences in the underlying population length structure. For example, in low selectivity scenarios for gummy sharks \textasciitilde60\% of individuals were immature compared to \textasciitilde50\% in the corresponding scenarios for school sharks (Figure S9). As \(P_{Max}\) decreased the proportion of maternal females in the analysis also decreased.

Overall, the 2PLF-maternity method performed the poorest. Ignoring \(P_{Max}\) introduced a fixed bias in this parameter that increased in magnitude as reproductive periodicity increased (Figure S3). This manifested in an overestimation of \(L'_{50}\) that was exacerbated by gear selectivity effects (Figure S2). In contrast, using a maturity function to approximate the maternity function (2PLF-maturity method) resulted in relatively good performance. \(L_{50}\) could usually be estimated with a higher precision than \(L'_{50}\). For the gummy shark, where \(L_{50}\) and \(L'_{50}\) were relatively similar, using the 2PLF-maturity method led to a constant underestimate of \(L'_{50}\) of approximately 2\%, exceeding the performance of the 3PLF-methods at most sample sizes.

The effect of the different methods in ultimately calculating \(R_0\) varied considerably across the various scenarios. For school sharks, estimating \(P_{Max}\), was the most effective way to minimize bias in most scenarios (Figure 5, Figure S10). Substituting maturity parameters (2PLF-maturity method) led to a median overestimation of \(R_0\) by 3.4\% across all scenarios, while the 2PLF-maternity method led to a median underestimate of 11.1\%. (Table S1, Figure S10). In contrast, for the gummy shark substituting maturity parameters led to a median overestimation of \(R_0\) of just 1\% across all scenarios which outperformed the 3PLF methods in several instances (Figure 5, Figure S10). Using the 2PLF method for gummy sharks led to a median overestimate of \(R_0\) of 22\% (Figure S10). For both species, better precision in calculating \(R_0\) was achieved with the 3PLF methods, with the 3PLF-estimated method performing best in most scenarios tested (Figure 6, Figure S11). For gummy sharks the 3PLF-fixed method performed best at lower sample sizes and lower values of \(P_{Max}\).

\subsection{Empirical case study}\label{empirical-case-study-1}

Maternal data were re-analysed for 1087 sandbar sharks including 0 mature individuals of which \ensuremath{\infty{}}\% were in maternal condition. Using the 3PLF-estimated method, the maximum likelihood estimate for \(\hat{P_{Max}}\) was 0.48 (Table 2). Despite having a sample size of \textgreater{} 600 mature females, the proportion of maternal individuals at length was still uncertain and \(\hat{P_{Max}}\) was estimated to lie between 0.39 and 0.60 with 95\% confidence. For 3PLF-fixed models a value of \(P_{Max} = 0.5\) was strongly supported (\(\Delta_{AIC}\) = 14.15) over the alternative fixed value of \(P_{Max} = 0.33\). The model with \(P_{Max} = 0.5\) also outperformed the 3PLF-estimated method (\(\Delta_{AIC}\) = 1.86). Given the fixed model had one fewer estimated parameters, both models had essentially the same level of support given the data (Burnham and Anderson 2002).

\section{Discussion}\label{discussion}

Relatively few reproductive biology studies have used maternity functions to model maternal reproductive output in chondrichthyan fishes. Where they have been employed the approach has typically been to use a three parameter logistic function with a fixed, user-defined value for the upper asymptote, \(P_{Max}\). Here we show that it is feasible to estimate \(P_{Max}\) from maternal data, in turn enabling statistical inferences on reproductive periodicity. Applying 3PLF models with estimated and fixed values of \(P_{Max}\) to simulated data showed that precision, bias, and confidence interval coverage often improved when \(P_{Max}\) was estimated. Using a fixed value for \(P_{Max}\) in some cases resulted in lower bias at low sample sizes. This study also demonstrated that recruitment can be overestimated when maturity data were used to approximate maternal data and inaccurate if \(P_{Max}\) was ignored when estimating maternity parameters. Based on these findings we outline considerations for practitioners using these methods and illustrate how they can provide novel insights into reproductive biology. We conclude by discussing the advantages of adopting this approach to quantifying maternal reproductive output and future directions.

\subsection{Implementing maternity functions}\label{implementing-maternity-functions}

Using simulated data to compare the relative performance of the 3PLF-estimated and 3PLF-fixed methods subject to a range of variables showed that it was feasible to estimate \(P_{Max}\) from data, but also identified situations where it may be preferable to fix \(P_{Max}\). In most scenarios sample sizes of at least 100 were needed for the 3PLF-estimated method to approach or exceed the performance of the 3PLF-fixed method. At sample sizes below this it is therefore advisable to fix \(P_{Max}\). Larger sample sizes are desirable before attempting to estimate \(P_{Max}\) in species with a triennial or longer reproductive cycle, due to the decline in the performance of the 3PLF-estimated method as \(P_{Max}\) decreases.

While the estimation of maternity parameters in most simulations was possible, it was difficult to do so accurately. Outcomes of the simulation study suggested that samples sizes of \(\geq500\) would be needed to estimate \(P_{Max}\) with \textless10\% MARE for a triennially reproducing species. This is a much larger error than is achievable from conventional logistic maturity analysis (Roa et al. 1999). In the case of the gummy shark this imprecision led to the 2PLF-maturity method performing comparably or better than the 3PLF methods in ultimately quantifying \(R_0\). Relatively wide confidence intervals were also obtained for \(P_{Max}\) in the empirical analysis for sandbar sharks, even with a sample size of \textgreater{} 1000. A similar sample size was used by Colonello \emph{et al}. (2016) to successfully estimate \(P_{Max}\) and maternity parameters for south Atlantic spiny dogfish. These results suggest that data requirements of the 3PLF analyses may be prohibitively large for many chondrichthyans and therefore best suited to use on commercially captured species where large sample sizes can be obtained (Oddone et al. 2010, Tribuzio and Kruse 2012).

In light of the data requirements suggested by this study, the current practice of using maturity parameters as a proxy for maternity parameters will likely still be the only option for numerous data-poor chondrichthyans. From this perspective, the outperformance of the 3PLF-methods by the 2PLF-methods for the gummy shark in several simulations is encouraging. However, the extent to which maturity parameters can provide a good approximation of maternity parameters may be species-specific, depending on how close \(L_{50}\) is to \(L'_{50}\). While several studies have shown these parameters to be similar (Walker 2007, Soto-López et al. 2018), \(L'_{50}\) is more frequently shifted to the right of \(L_{50}\) (Braccini et al. 2006, Montealegre-Quijano et al. 2014, Colonello et al. 2016, Palacios-Hernández et al. 2020), and this length difference can equate to one or more years. For example, Harry \emph{et al.} (2013) found both spot-tail shark, \emph{C. sorrah}, and Australian blacktip shark, \emph{C. tilstoni}, began reproducing the year after reaching sexual maturity. Similarly, Fujiyama found \(A'_{50}\) to be 1.4 years older than \(A_{50}\) in blue sharks.

More research is needed to understand the relationship between size at maturity and maternity and variability within and among taxa. In general, this study confirms Walker's (2005) assertion that the common practice of weighting the maturity curve by the frequency of parturition overestimates recruitment. This study also confirms that fitting a 2PLF to maternal data when \(P_{Max} < 1\) (effectively ignoring \(P_{Max}\)) is likely to result in biased parameters and can also overestimate recruitment. While not widely used, this approach has been undertaken in several studies (Baremore and Hale 2012, Mejía‐Falla et al. 2012, Baremore and Passerotti 2013, Rambahiniarison et al. 2018).

The success of any analysis using the 3PLF-fixed or 2PLF-maturity methods, as implemented here, ultimately depends on the analyst choosing a suitable value for \(P_{Max}\). In this study, the value for \(P_{Max}\) used in the fixed analyses was chosen based on the proportion of maternal females observed in the simulated data. The potential to introduce a greater level of bias in the analysis through the incorrect selection of \(P_{Max}\) should also be considered if using this approach.

\subsection{Empirical study}\label{empirical-study}

The empirical analysis carried out using data for the sandbar shark illustrates how novel insights can be gained from maternal data by estimating \(P_{Max}\). Detailed reproductive studies of the western North Atlantic population by Baremore and Hale (2012) and Piercy \emph{et al.} (2016) were ambiguous about the frequency of reproduction. In both studies, the low proportion of pregnant females and bimodality in ovarian follicle size led the authors to conclude that the reproductive cycle was most commonly triennial or possibly longer at the population level. Reanalysis of these combined data sets using the 3PLF-estimated method, however, resulted in a maximum likelihood estimate of \(\hat{P_{Max}} = 0.48\), considerably larger than 0.37 - the proportion of pregnant mature females observed by Baremore and Hale (2012). Similarly, based on AIC values, a fixed value of \(P_{Max} = 0.5\) had a much higher relative plausibility than a value of 0.33 (Table 2).

This unexpected result may be related to the gradual attainment of asymptotic maternal status. Maturity in sandbar sharks appears to occur over an extended size range. \(L_{5}\) and \(L_{95}\) are 146cm and 176cm -- a length interval that corresponds to around seven years given the sandbar shark's slow growth rate (Hale and Baremore 2013). With females maturing over a broad range of sizes and ages, it is possible that smaller females reproduce less frequently. The outcomes of this study suggest that larger sharks may ultimately be capable of reproducing biennially, although few females appear to attain this size (Hale and Baremore 2013). Further work would be required to confirm this hypothesis, however, such an outcome would have important implications for management; not only are larger females more fecund (Baremore and Hale 2012), they also reproduce more frequently. This implies that size selective fishing or the removal of larger and older female sharks could have a disproportionately greater impact on population productivity.

\subsection{Advantages and future directions}\label{advantages-and-future-directions}

Despite the simplicity of the 3PLF-estimated method, it nonetheless represents a conceptual shift for chondrichthyan reproductive and population biology. Until now reproductive periodicity has largely been determined qualitatively from observations of female reproductive biology or inferred based on circumstantial evidence. Using the 3PLF-estimated method to estimate \(P_{Max}\) within a maternity function transforms it from a nuisance parameter to one of direct inferential interest that can be seen as a valuable output of the modelling process itself. In addition to the potential benefits of using this method to improve estimation of maternal parameters, there are also a range of other advantages that arise from estimating \(P_{Max}\).

One benefit of this approach is that it that it reduces the need for subjective modelling assumptions relating to \(P_{Max}\). Such assumptions can have important implications for population modelling. For example, to account for uncertainty in temporal and spatial frequency of reproduction in sparsely-spotted stingarees, \emph{Urolophus paucimaculatus}, Trinnie \emph{et al}. (Trinnie et al. 2014) estimated maternity parameters for ten plausible scenarios. In stock assessment of western North Atlantic sandbar sharks a breeding frequency of 2.5 years was assumed to account for uncertainty in the duration of the reproductive cycle (SEDAR 2017).

More commonly, practitioners have chosen fixed values of the form \(P_{Max} = 1/n\), where \emph{n} is the apparent duration of the reproductive cycle in years. This process also has the potential to introduce bias if a small proportion of females reproduce more or less frequently that the larger population. Long-term study of offspring from genetically profiled lemon sharks, \emph{Negaprion brevirostris}, confirmed that most individuals reproduced biennially, but also revealed some cases of triennial reproduction (Feldheim et al. 2014). Statistical estimation of \(P_{Max}\) from data avoids these decisions and enables uncertainty in this parameter to be included in subsequent population models.

Estimation of \(P_{Max}\) using the 3PLF-estimated method may also be a means to resolving the longstanding question of reproductive frequency in some rare and threatened chondrichthyans. White sharks, \emph{Carcharodon carcharias}, and whale shark, \emph{Rhinchodon typus}, are two examples of intensively studied species where only fragmentary observations of female reproductive biology exist (Joung et al. 1996, Sato et al. 2016). In both species lack of data on reproductive frequency is a major impediment to population modelling and the development of management strategies (Huveneers et al. 2018, Bowlby and Gibson 2020). With many of these populations the focus of ongoing monitoring, non-lethal methods for assessing maternal state may eventually provide a way of collecting sufficient data for estimation of \(P_{Max}\) (Sulikowski et al. 2016).

A logical progression from estimating \(P_{Max}\) is the consideration of alternative functional forms for this parameter such as time- or space-varying \(P_{Max}\) or more complex length- or age-dependent forms (Winton et al. 2014). Despite the diverse range of reproductive modes found in chondrichthyan fishes, the current, limited knowledge of maternal investment has typically investigated variables such as litter size and pup condition (Hussey et al. 2010). Little is known about how reproductive frequency varies as a function of size or age, in response to environmental effects, or at changing population densities. Better understanding of these factors has the potential to improve understanding reproductive strategies and is, in many cases, of direct relevance to management.

\section{Acknowledgements}\label{acknowledgements}

We thank John Carlson and all NMFS staff who were involved in collecting the sandbar shark reproductive data used in this study. We also thank Alex Hesp for valuable comments on an early draft of this paper and generosity in sharing R code.

\section{Competing interests}\label{competing-interests}

The authors declare there are no competing interests.

\section{Data availability}\label{data-availability}

Data generated or analyzed during this study are available from the corresponding author upon reasonable request.

\section*{References}\label{references}
\addcontentsline{toc}{section}{References}

\phantomsection\label{refs}
\begin{CSLReferences}{1}{0}
\bibitem[\citeproctext]{ref-baremore_reproduction_2012}
Baremore, I.E., and Hale, L.F. 2012. Reproduction of the {Sandbar} {Shark} in the {Western} {North} {Atlantic} {Ocean} and {Gulf} of {Mexico}. Marine and Coastal Fisheries \textbf{4}(1): 560--572. doi: \href{https://doi.org/10.1080/19425120.2012.700904}{10.1080/19425120.2012.700904}.

\bibitem[\citeproctext]{ref-baremore_reproduction_2013}
Baremore, I.E., and Passerotti, M.S. 2013. Reproduction of the {Blacktip} {Shark} in the {Gulf} of {Mexico}. Marine and Coastal Fisheries \textbf{5}(1): 127--138. doi: \href{https://doi.org/10.1080/19425120.2012.758204}{10.1080/19425120.2012.758204}.

\bibitem[\citeproctext]{ref-bowlby_implications_2020}
Bowlby, H.D., and Gibson, A.J.F. 2020. Implications of life history uncertainty when evaluating status in the {Northwest} {Atlantic} population of white shark (\emph{{Carcharodon} carcharias)}. Ecology and Evolution \textbf{10}(11): 4990--5000. doi: \href{https://doi.org/10.1002/ece3.6252}{10.1002/ece3.6252}.

\bibitem[\citeproctext]{ref-braccini_determining_2006}
Braccini, J.M., Gillanders, B.M., and Walker, T.I. 2006. Determining reproductive parameters for population assessments of chondrichthyan species with asynchronous ovulation and parturition: Piked spurdog (\emph{{Squalus} megalops}) as a case study. Mar. Freshwater Res. \textbf{57}: 105--119. doi: \href{https://doi.org/doi:\%2010.1071/Mf05076}{doi: 10.1071/Mf05076}.

\bibitem[\citeproctext]{ref-burnham_model_2002}
Burnham, K.P., and Anderson, D.R. 2002. Model selection and multimodel inference: A practical information-theoretic approach. 2nd edn. Springer, New York, NY.

\bibitem[\citeproctext]{ref-colonello_reproductive_2016}
Colonello, J.H., Cortés, F., Belleggia, M., and Massa, A.M. 2016. Reproductive and population parameters of spiny dogfish \emph{{Squalus} acanthias} in the south-western {Atlantic} {Ocean}. J. Fish Biol. \textbf{88}(5): 1758--1775. doi: \href{https://doi.org/10.1111/jfb.12940}{10.1111/jfb.12940}.

\bibitem[\citeproctext]{ref-cortes_demographic_1998}
Cortes, E. 1998. Demographic analysis as an aid in shark stock assessment and management. Fish. Res. \textbf{39}: 199--208. doi: \href{https://doi.org/10.1016/S0165-7836(98)00183-0}{10.1016/S0165-7836(98)00183-0}.

\bibitem[\citeproctext]{ref-cortes_population_2012}
Cortés, E., Brooks, E.N., and Gedamke, T. 2012. Population dynamics, demography, and stock assessment. \emph{In} Biology of {Sharks} and {Their} {Relatives}, {Second} {Edition}. \emph{Edited by} J.C. Carrier, J.A. Musick, and M.R. Heithaus. CRC Press, New York. pp. 453--86.

\bibitem[\citeproctext]{ref-dulvy_you_2008}
Dulvy, N.K., Baum, J.K., Clarke, S., Compagno, L.J.V., Cortes, E., Domingo, A., Fordham, S., Fowler, S., Francis, M.P., Gibson, C., Martínez, J., Musick, J.A., Soldo, A., Stevens, J.D., and Valenti, S. 2008. You can swim but you can't hide: The global status and conservation of oceanic pelagic sharks and rays. Aquatic Conservation: Marine and Freshwater Ecosystems.

\bibitem[\citeproctext]{ref-feldheim_two_2014}
Feldheim, K.A., Gruber, S.H., DiBattista, J.D., Babcock, E.A., Kessel, S.T., Hendry, A.P., Pikitch, E.K., Ashley, M.V., and Chapman, D.D. 2014. Two decades of genetic profiling yields first evidence of natal philopatry and long-term fidelity to parturition sites in sharks. Mol. Ecol. \textbf{23}(1): 110--117. doi: \href{https://doi.org/10.1111/mec.12583}{10.1111/mec.12583}.

\bibitem[\citeproctext]{ref-frisk_life_2005}
Frisk, M.G., Miller, T.J., and Dulvy, N.K. 2005. Life histories and vulnerability to exploitation of elasmobranchs: {Inferences} from elasticity, pertubation and phylogenetic analyses. Journal of Northwest Atlantic Fishery Science \textbf{35}: 27--45. doi: \href{https://doi.org/10.2960/J.v35.m514}{10.2960/J.v35.m514}.

\bibitem[\citeproctext]{ref-fujinami_reproductive_2017}
Fujinami, Y., Semba, Y., Okamoto, H., Ohshimo, S., and Tanaka, S. 2017. Reproductive biology of the blue shark (\emph{{Prionace} glauca}) in the western {North} {Pacific} {Ocean}. Mar. Freshwater Res. \textbf{68}(11): 2018--2027. Available from \url{https://doi.org/10.1071/MF16101}.

\bibitem[\citeproctext]{ref-grant_estimation_1979}
Grant, C., Sandland, R., and Olsen, A. 1979. Estimation of growth, mortality and yeild per recruit of the {Australian} school shark, \emph{{Galeorhinus} australis} ({Macleay}), from tag recoveries. Mar. Freshwater Res. \textbf{30}(5): 625--637. doi: \href{https://doi.org/10.1071/MF9790625}{10.1071/MF9790625}.

\bibitem[\citeproctext]{ref-hale_age_2013}
Hale, L.F., and Baremore, I.E. 2013. Age and {Growth} of the {Sandbar} {Shark} (\emph{{Carcharhinus} plumbeus}) from the {Northern} {Gulf} of {Mexico} and the {Western} {North} {Atlantic} {Ocean}. Gulf of Mexico Science \textbf{1}(2): 28--39.

\bibitem[\citeproctext]{ref-hall_modelling_2000}
Hall, N., Cao, L., Chubb, C., Caputi, N., Cheng, H., Melville-Smith, R., and Shanks, S. 2000. Modelling to explore management strategies to optimise the value of the {Rock} {Lobster} {Fishery} of {Western} {Australia}. Fisheries Research Development Corporation Final Report Project (97/104).

\bibitem[\citeproctext]{ref-harry_age_2013}
Harry, A.V., Tobin, A.J., and Simpfendorfer, C.A. 2013. Age, growth and reproductive biology of the spot-tail shark,\emph{{Carcharhinus} sorrah}, and the {Australian} blacktip shark, \emph{{Carcharhinus} tilstoni}, from the {Great} {Barrier} {Reef} {World} {Heritage} {Area}, north-eastern {Australia}. Mar. Freshwater Res. \textbf{64}: 277--293. doi: \href{https://doi.org/10.1071/MF12142}{10.1071/MF12142}.

\bibitem[\citeproctext]{ref-hesp_l3assess_2023}
Hesp, A. 2023. {L3Assess}: {Catch} curve and per recruit analyses. R package version 0.1.0.

\bibitem[\citeproctext]{ref-higgs_reproductive_2020}
Higgs, J.M., Hoffmayer, E.R., Sulikowski, J.A., Driggers, W.B., Stiller, D.A., and Hendon, J.M. 2020. Reproductive biology of the finetooth shark (\emph{{Carcharhinus} isodon}) in the northern {Gulf} of {Mexico}, with evidence of both annual and biennial reproduction. Mar. Freshwater Res. Available from \url{https://doi.org/10.1071/MF20120}.

\bibitem[\citeproctext]{ref-hilborn1992}
Hilborn, R., and Walters, C.J. 1992. Quantitative fisheries stock assessment. Choice, dynamics and uncertainty. Springer, New York, NY.

\bibitem[\citeproctext]{ref-hussey_maternal_2010}
Hussey, N.E., Wintner, S.P., Dudley, S.F.J., Cliff, G., Cocks, D.T., and MacNeil, M.A. 2010. Maternal investment and size-specific reproductive output in carcharhinid sharks. J. Anim. Ecol. \textbf{79}: 184--193. doi: \href{https://doi.org/doi:\%2010.1111/j.1365-2656.2009.01623.x}{doi: 10.1111/j.1365-2656.2009.01623.x}.

\bibitem[\citeproctext]{ref-huveneers_future_2018}
Huveneers, C., Apps, K., Becerril-García, E.E., Bruce, B., Butcher, P.A., Carlisle, A.B., Chapple, T.K., Christiansen, H.M., Cliff, G., Curtis, T.H., Daly-Engel, T.S., Dewar, H., Dicken, M.L., Domeier, M.L., Duffy, C.A.J., Ford, R., Francis, M.P., French, G.C.A., Galván-Magaña, F., García-Rodríguez, E., Gennari, E., Graham, B., Hayden, B., Hoyos-Padilla, E.M., Hussey, N.E., Jewell, O.J.D., Jorgensen, S.J., Kock, A.A., Lowe, C.G., Lyons, K., Meyer, L., Oelofse, G., Oñate-González, E.C., Oosthuizen, H., O'Sullivan, J.B., Ramm, K., Skomal, G., Sloan, S., Smale, M.J., Sosa-Nishizaki, O., Sperone, E., Tamburin, E., Towner, A.V., Wcisel, M.A., Weng, K.C., and Werry, J.M. 2018. Future research directions on the {``{Elusive}''} white shark. Frontiers in Marine Science \textbf{5}: 455. doi: \href{https://doi.org/10.3389/fmars.2018.00455}{10.3389/fmars.2018.00455}.

\bibitem[\citeproctext]{ref-huveneers_reproductive_2007}
Huveneers, C., Walker, T.I., Otway, N.M., and Harcourt, R.G. 2007. Reproductive synchrony of three sympatric species of wobbegong shark (genus \emph{orectolobus}) in {New} {South} {Wales}, {Australia}: Reproductive parameter estimates necessary for population modelling. Mar. Freshwater Res. \textbf{58}: 765--777. doi: \href{https://doi.org/10.1071/MF06187}{10.1071/MF06187}.

\bibitem[\citeproctext]{ref-jennings_marine_2001}
Jennings, S., Kaiser, M., and Reynolds, J.D. 2001. Marine fisheries ecology. John Wiley \& Sons, Oxford.

\bibitem[\citeproctext]{ref-joung_whale_1996}
Joung, S.-J., Chen, C.-T., Clark, E., Uchida, S., and Huang, W.Y.P. 1996. The whale shark, \emph{{Rhincodon} typus}, is a livebearer: 300 embryos found in one {``megamamma''} supreme. Environ. Biol. Fish. \textbf{46}(3): 219--223. doi: \href{https://doi.org/10.1007/BF00004997}{10.1007/BF00004997}.

\bibitem[\citeproctext]{ref-king_fisheries_2007}
King, M. 2007. Fisheries biology, assessment and management. Blackwell Publishing, Oxford.

\bibitem[\citeproctext]{ref-kirkwood_gill_1986}
Kirkwood, G.P., and Walker, T.I. 1986. Gill net mesh selectivities for gummy shark, \emph{{Mustelus} antarcticus} {Gunther}, taken in southeastern {Australian} waters. Aust. J. Mar. Fresh. Res. \textbf{37}: 689--697.

\bibitem[\citeproctext]{ref-kristensen_tmb:_2016}
Kristensen, K., Nielsen, A., Berg, C.W., Skaug, H., and Bell, B.M. 2016. {TMB}: {Automatic} {Differentiation} and {Laplace} {Approximation}. J. Stat. Softw. \textbf{70}(5): 1--21. doi: \href{https://doi.org/10.18637/jss.v070.i05}{10.18637/jss.v070.i05}.

\bibitem[\citeproctext]{ref-mejiafalla_reproductive_2012}
Mejía‐Falla, P.A., Navia, A.F., and Cortés, E. 2012. Reproductive variables of \emph{{Urotrygon} rogersi} ({Batoidea}: {Urotrygonidae}): A species with a triannual reproductive cycle in the eastern tropical {Pacific} {Ocean}. J. Fish Biol. \textbf{80}(5): 1246--1266. doi: \href{https://doi.org/10.1111/j.1095-8649.2012.03237.x}{10.1111/j.1095-8649.2012.03237.x}.

\bibitem[\citeproctext]{ref-mollet_comparative_2002}
Mollet, H.F., and Cailliet, G.M. 2002. Comparative population demography of elasmobranchs using life history tables, {Leslie} matrices and stage-based matrix models. Mar. Freshwater Res. \textbf{53}: 503--516.

\bibitem[\citeproctext]{ref-montealegre-quijano_sexual_2014}
Montealegre-Quijano, S., Cardoso, A.T.C., Silva, R.Z., Kinas, P.G., and Vooren, C.M. 2014. Sexual development, size at maturity, size at maternity and fecundity of the blue shark \emph{{Prionace} glauca} ({Linnaeus}, 1758) in the {Southwest} {Atlantic}. Fish. Res. \textbf{160}: 18--32. doi: \url{https://doi.org/10.1016/j.fishres.2014.03.003}.

\bibitem[\citeproctext]{ref-motulsky_fitting_2004}
Motulsky, H., and Christopoulos, A. 2004. Fitting models to biological data using linear and nonlinear regression: A practical guide to curve fitting. Oxford University Press, San Diego.

\bibitem[\citeproctext]{ref-moulton_age_1992}
Moulton, P.L., Walker, T.I., and Saddlier, S.R. 1992. Age and growth-studies of gummy shark, \emph{{Mustelus} antarcticus} {Gunther}, and school shark, \emph{{Galeorhinus} galeus}({Linnaeus}), from {Southern} {Australian} waters. Aust. J. Mar. Fresh. Res. \textbf{43}: 1241--1267. doi: \href{https://doi.org/10.1071/MF9921241}{10.1071/MF9921241}.

\bibitem[\citeproctext]{ref-oddone_size_2010}
Oddone, M.C., Paesch, L., and Norbis, W. 2010. Size structure, abundance and preliminary information on the reproductive parameters of the shortspine spurdog (\emph{{Squalus} mitsukurii}) in the argentinean-uruguayan common fishing zone from the mid-1990s. J. Northw. Atl. Fish. Sci. \textbf{43}: 13--26. doi: \href{https://doi.org/10.2960/j.v.43.m662}{10.2960/j.v.43.m662}.

\bibitem[\citeproctext]{ref-palacios-hernandez_temporal_2020}
Palacios-Hernández, D., Castillo-Géniz, J.L., Méndez-Loeza, I., and Pérez-Jiménez, J.C. 2020. Temporal and latitudinal comparisons of reproductive parameters in a heavily exploited shark, the bonnethead, \emph{{Sphyrna} tiburo} ({L}. 1758), in the southern {Gulf} of {Mexico}. J. Fish Biol. \textbf{97}(1): 100--112. doi: \href{https://doi.org/10.1111/jfb.14330}{10.1111/jfb.14330}.

\bibitem[\citeproctext]{ref-piercy_histological_2016}
Piercy, A.N., Murie, D.J., and Gelsleichter, J.J. 2016. Histological and morphological aspects of reproduction in the sandbar shark \emph{{Carcharhinus} plumbeus} in the {U}.{S}. South-eastern {Atlantic} {Ocean} and {Gulf} of {Mexico}. J. Fish Biol. \textbf{88}(5): 1708--1730. doi: \href{https://doi.org/10.1111/jfb.12945}{10.1111/jfb.12945}.

\bibitem[\citeproctext]{ref-punt_estimating_1997}
Punt, A.E., Kennedy, R.B., and Frusher, S.D. 1997. Estimating the size-transition matrix for {Tasmanian} rock lobster, \emph{{Jasus} edwardsii}. Mar. Freshwater Res. \textbf{48}(8): 981--992. Available from \url{https://doi.org/10.1071/MF97017}.

\bibitem[\citeproctext]{ref-punt_stock_1998}
Punt, A.E., and Walker, T.I. 1998. Stock assessment and risk analysis for the school shark (\emph{{Galeorhinus} galeus)} off southern {Australia}. Mar. Freshwater Res. \textbf{49}: 719--731. doi: \href{https://doi.org/10.1071/MF96101}{10.1071/MF96101}.

\bibitem[\citeproctext]{ref-quinn_quantitative_1999}
Quinn, I.I., and Deriso, R.B. 1999. Quantitative fish dynamics. Oxford University Press, New York, NY.

\bibitem[\citeproctext]{ref-r_core_team_r:_2022}
R Core Team. 2022. R: {A} {Language} and {Environment} for {Statistical} {Computing}. Vienna, Austria. Available from \url{https://www.R-project.org/}.

\bibitem[\citeproctext]{ref-rambahiniarison_life_2018}
Rambahiniarison, J.M., Lamoste, M.J., Rohner, C.A., Murray, R., Snow, S., Labaja, J., Araujo, G., and Ponzo, A. 2018. Life history, growth, and reproductive biology of four mobulid species in the {Bohol} {Sea}, {Philippines}. Frontiers in Marine Science \textbf{5}: 269. doi: \href{https://doi.org/10.3389/fmars.2018.00269}{10.3389/fmars.2018.00269}.

\bibitem[\citeproctext]{ref-rigby_patterns_2015}
Rigby, C., and Simpfendorfer, C.A. 2015. Patterns in life history traits of deep-water chondrichthyans. Deep Sea Research Part II: Topical Studies in Oceanography \textbf{115}: 30--40. doi: \url{https://doi.org/10.1016/j.dsr2.2013.09.004}.

\bibitem[\citeproctext]{ref-roa_estimation_1999}
Roa, R., Ernst, B., and Tapia, F. 1999. Estimation of size at sexual maturity: An evaluation of analytical and resampling procedures. Fish. B-noaa. \textbf{97}: 570--580.

\bibitem[\citeproctext]{ref-rochowski_reproductive_2015}
Rochowski, B.E.A., Graham, K.J., Day, R.W., and Walker, T.I. 2015. Reproductive biology of the greeneye spurdog \emph{{Squalus} chloroculus} ({Squaliformes}, {Squalidae}). J. Fish Biol. \textbf{86}(2): 734--754. doi: \href{https://doi.org/10.1111/jfb.12593}{10.1111/jfb.12593}.

\bibitem[\citeproctext]{ref-rudd_accounting_2018}
Rudd, M.B., and Thorson, J.T. 2018. Accounting for variable recruitment and fishing mortality in length-based stock assessments for data-limited fisheries. Can. J. Fish. Aquat. Sci. \textbf{75}(7): 1019--1035.

\bibitem[\citeproctext]{ref-sato_how_2016}
Sato, K., Nakamura, M., Tomita, T., Toda, M., Miyamoto, K., and Nozu, R. 2016. How great white sharks nourish their embryos to a large size: Evidence of lipid histotrophy in lamnoid shark reproduction. Biology Open \textbf{5}(9): 1211--1215. doi: \href{https://doi.org/10.1242/bio.017939}{10.1242/bio.017939}.

\bibitem[\citeproctext]{ref-sedar_sedar_2012}
SEDAR. 2012. {SEDAR} 29 stock assessment report: {HMS} {Gulf} of {Mexico} blacktip shark. North Charleston, SC.

\bibitem[\citeproctext]{ref-sedar_sedar_2017}
SEDAR. 2017. {SEDAR} 54 stock assessment report: {HMS} sandbar shark. North Charleston, SC.

\bibitem[\citeproctext]{ref-sedar_sedar_2023}
SEDAR. 2023. {HMS} {Hammerhead} {Sharks}: {Scalloped} {Hammerhead} {Shark}. {Section} {III}: {Assessment} {Report}. North Charleston, SC.

\bibitem[\citeproctext]{ref-sminkey_age_1995}
Sminkey, T.R., and Musick, J.A. 1995. Age and growth of the sandbar shark, \emph{{Carcharhinus} plumbeus}, before and after population depletion. Copeia: 871--883.

\bibitem[\citeproctext]{ref-soto-lopez_reproductive_2018}
Soto-López, K., Ochoa-Báez, R.I., Tovar-Ávila, J., and Galván-Magaña, F. 2018. Reproductive biology of the brown smooth-hound shark, \emph{{Mustelus} henlei} ({Chondrichthyes}: {Triakidae}), off northwestern {Mexico} based on macroscopic and histological analyses. Cienc. Mar. \textbf{44}(2): 125--139.

\bibitem[\citeproctext]{ref-springer_natural_1960}
Springer, S. 1960. Natural history of the sandbar shark, \emph{{Eulamia} milberti}. US Fish Wildl. Serv. Fish. Bull. \textbf{61}: 1--38.

\bibitem[\citeproctext]{ref-stevens_variable_1999}
Stevens, J.D. 1999. Variable resilience to fishing pressure in two sharks: The significance of different ecological and life history parameters. Am. Fish. S. S. \textbf{23}: 11--15.

\bibitem[\citeproctext]{ref-sulikowski_seasonal_2016}
Sulikowski, J.A., Wheeler, C.R., Gallagher, A.J., Prohaska, B.K., Langan, J.A., and Hammerschlag, N. 2016. Seasonal and life-stage variation in the reproductive ecology of a marine apex predator, the tiger shark \emph{{Galeocerdo} cuvier}, at a protected female-dominated site. Aquat. Biol. \textbf{24}(3): 175--184.

\bibitem[\citeproctext]{ref-taylor_stockrecruitment_2013}
Taylor, I.G., Gertseva, V., Methot, R.D., and Maunder, M.N. 2013. A stock--recruitment relationship based on pre-recruit survival, illustrated with application to spiny dogfish shark. Fish. Res. \textbf{142}: 15--21. doi: \url{https://doi.org/10.1016/j.fishres.2012.04.018}.

\bibitem[\citeproctext]{ref-taylor_living_2016}
Taylor, S.M., Harry, A.V., and Bennett, M.B. 2016. Living on the edge: Latitudinal variations in the reproductive biology of two coastal species of sharks. J. Fish Biol. \textbf{89}(5): 2399--2418. doi: \href{https://doi.org/10.1111/jfb.13126}{10.1111/jfb.13126}.

\bibitem[\citeproctext]{ref-tribuzio_life_2012}
Tribuzio, C.A., and Kruse, G.H. 2012. Life history characteristics of a lightly exploited stock of \emph{{Squalus} suckleyi}. J. Fish Biol. \textbf{80}(5): 1159--1180. doi: \href{https://doi.org/10.1111/j.1095-8649.2012.03241.x}{10.1111/j.1095-8649.2012.03241.x}.

\bibitem[\citeproctext]{ref-trinnie_reproductive_2009}
Trinnie, F.I., Walker, T.I., Jones, P.L., and Laurenson, L.J. 2009. Reproductive biology of the eastern shovelnose stingaree \emph{{Trygonoptera} imitata} from south-eastern {Australia}. Mar. Freshwater Res. \textbf{60}: 845--860.

\bibitem[\citeproctext]{ref-trinnie2014}
Trinnie, F.I., Walker, T.I., Jones, P.L., and Laurenson, L.J. 2014. Regional differences in the reproductive parameters of the sparsely-spotted stingaree, {\emph{Urolophus paucimaculatus}}, from south-eastern Australia. Mar. Freshwater Res. \textbf{65}(11): 943. doi: \href{https://doi.org/10.1071/mf13275}{10.1071/mf13275}.

\bibitem[\citeproctext]{ref-trinnie_reproductive_2016}
Trinnie, F.I., Walker, T.I., Jones, P.L., and Laurenson, L.J. 2016. Reproductive cycle of \emph{{Urolophus} cruciatus} in south-eastern {Australia}: {Does} the species exhibit obligate or facultative diapause? Mar. Biol. \textbf{163}(11): 226. doi: \href{https://doi.org/10.1007/s00227-016-2976-x}{10.1007/s00227-016-2976-x}.

\bibitem[\citeproctext]{ref-walker_fishery_1992}
Walker, T. 1992. Fishery simulation model for sharks applied to the {Gummy} {Shark}, \emph{{Mustelus} antarcticus} {Gunther}, from {Southern} {Australian} waters. Mar. Freshwater Res. \textbf{43}(1): 195--212. Available from \url{https://doi.org/10.1071/MF9920195}.

\bibitem[\citeproctext]{ref-walker_reproduction_2005}
Walker, T.I. 2005. Reproduction in fisheries science. \emph{In} Reproductive biology and phylogeny of {Chondrichthyans}: Sharks, batoids, and chimaeras. \emph{Edited by} W.C. Hamlett. Science Publishers Inc., Enfield. pp. 81--127.

\bibitem[\citeproctext]{ref-walker_spatial_2007}
Walker, T.I. 2007. Spatial and temporal variation in the reproductive biology of gummy shark \emph{{Mustelus} antarcticus} ({Chondrichthyes} : {Triakidae}) harvested off southern {Australia}. Mar. Freshwater Res. \textbf{58}: 67--97. doi: \href{https://doi.org/10.1071/MF06074}{10.1071/MF06074}.

\bibitem[\citeproctext]{ref-winton2014}
Winton, M.V., Wuenschel, M.J., and McBride, R.S. 2014. Investigating spatial variation and temperature effects on maturity of female winter flounder ({\emph{Pseudopleuronectes americanus}}) using generalized additive models. Can. J. Fish. Aquat. Sci. \textbf{71}(9): 1279--1290. doi: \href{https://doi.org/10.1139/cjfas-2013-0617}{10.1139/cjfas-2013-0617}.

\bibitem[\citeproctext]{ref-xiao_demographic_2000}
Xiao, Y.S., and Walker, T.I. 2000. Demographic analysis of gummy shark (\emph{{Mustelus} antarcticus}) and school shark (\emph{{Galeorhinus} galeus}) off southern {Australia} by applying a generalized {Lotka} equation and its dual equation. Can. J. Fish. Aquat. Sci. \textbf{57}: 214--222. doi: \href{https://doi.org/10.1139/f99-224}{10.1139/f99-224}.

\bibitem[\citeproctext]{ref-zuur_analyzing_2007}
Zuur, A., Ieno, E.N., and Smith, G.M. 2007. Analyzing ecological data. New York, NY.

\end{CSLReferences}

\end{document}
